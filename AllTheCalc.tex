\documentclass[letterpaper,normalheadings,twocolumn]{scrreprt}
%\documentclass[12pt,letterpaper]{scrreprt}

% Page Layout
% \usepackage{fullpage}
% \usepackage{multicol}
\usepackage[margin=0.5in]{geometry}

% Use utf-8 encoding for foreign characters
\usepackage[utf8]{inputenc}

% More symbols
\usepackage{graphicx,amsmath,amssymb,dsfont}

% Algorithms
\usepackage[ruled]{algorithm2e}

% Theorems
\usepackage{amsthm}
\newtheorem{definition}{Definition}
\newtheorem{theorem}{Theorem}
\newtheorem{lemma}{Lemma}
%\renewcommand{\thesection}{\arabic{section}}
%\renewcommand{\thesubsection}{\arabic{section}.\arabic{subsection}} 

% Package for including code in the document
\usepackage{listings}

% Captions
\usepackage[small,bf,hang]{caption}
\setlength{\captionmargin}{12pt}

\usepackage{verbatim}
\title{All The Calculus}

\begin{document}
\maketitle
%%%%%%%%%%%%%%%%%%%%%%%%%%%
%%%%%%%%%%%TRIG%%%%%%%%%%%%
%%%%%%%%%%%%%%%%%%%%%%%%%%%

\chapter{Trigonometric Identities}
\[\begin{array}{ccccc}
\sin^2{\theta} &+& \cos^2{\theta} &=& 1 \\
\tan^2{\theta} &+& 1 &=& \sec^2{\theta} \\
1 &+& \cot^2{\theta} &=& \csc^2{\theta} 
\end{array}\]
\[\begin{array}{ccc}
SOH & CAH & TOA \\
\sin\theta = \frac{opposite}{hypoteneuse} & \cos\theta = \frac{adjacent}{hypoteneuse} & \tan\theta = \frac{opposite}{adjacent} \\
\end{array}\]

%%%%%%%%%%%%%%%%%%%%%%%%%%%
%%%%%%%DERIVATIVES%%%%%%%%%
%%%%%%%%%%%%%%%%%%% \\%%%%%%%%

\chapter{Derivatives}

Definition \\
Product Rule \\
Chain Rule \\

\begin{table}[ht]
  \centering
  \begin{tabular}{|l|l||l|l|}
  \hline
  $f(u)$ & $\frac{d}{du}f(u)$ & $f^{-1}(u)$ & $\frac{d}{du}f^{-1}(u)$ \\  
  \hline
  & & & \\ $\sin{u}$ & $du\cos{u}$ & $\sin^{-1}{u}$ & $\frac{du}{\sqrt{1 - u^2}}$\\
  & & & \\ $\cos{u}$ & $-du\sin{u}$ & $\cos^{-1}{u}$ & $\frac{-du}{\sqrt{1 - u^2}}$\\
  & & & \\ $\tan{u}$ & $du\sec^2{u}$ & $\tan^{-1}{u}$ & $\frac{du}{\sqrt{1 + u^2}}$\\
  & & & \\ $\csc{u}$ & $-du\cot{u}\csc{u}$ & $\csc^{-1}{u}$ & $\frac{-du}{\sqrt{1-\frac{1}{u^2}} u^2}$\\
  & & & \\ $\sec{u}$ & $du\tan{u}\sec{u}$ & $\sec^{-1}{u}$ & $\frac{du}{\sqrt{1-\frac{1}{u^2}} u^2}$\\
  & & & \\ $\cot{u}$ & $-du\csc^2{u}$ & $\cot^{-1}{u}$ & $\frac{-du}{\sqrt{1 + u^2}}$\\ 
  & & & \\
  \hline
  \end{tabular}
  \caption{Trig Derivatives; Note that relations that hold for tan and sec also hold for cot and csc, but with a negative sign}
  \label{tab:results}
\end{table}


%%%%%%%%%%%%%%%%%%%%%%%%%%%
%%%%%%%%INTEGRALS%%%%%%%%%%
%%%%%%%%%%%%%%%%%%%%%%%%%%%

\chapter{Integration}
The \emph{integral} of a function $f(x)$ can be defined in a few ways. The simplest one to work with is that the integral $F(x)$ of $f(x)$ is the function that satisfies $F'(x) = f(x)$. It should perhaps be noted that $F(x)$ is perhaps more accurately a \emph{set} of functions, as derivation loses a certain amount of information (namely, constants). This is why we include the \emph{constant of integration} $C$ to the integral $F$, to indicate this loss of information. Using this definition we can think of finding the integral $F$ as figuring out what function was differentiated into $f$, which fits nicely with how we tend to approach integration. The integral represents the area under the curve $f(x)$ (with respect to the x-axis), but because of the constant of integration it therefore isn't especially meaningful to consider a \emph{single} value of $F(x)$, unless we are treating the integral simply as the anti-derivative, in which case we drop the constant of integration.
\section{Basics}%%%%%%%%%%%%%%%%%%%%%%%%%%%%%%%%%%%%%%%%%%%%%%%%%%%%%%%%%%%%%%%%%%%%%%%%%%%%%%%%%%%%%%%%%%%%%%%%%%%%%%%%%%%%%%%%%%%%%%%%%%%%%%%%
$\int{f(x) + g(x)} = \int{f(x)} + \int{g(x)}$ (integration is distributive over addition)
$\int{cf(x)} = c\int{f(x)}$ (integration is not effected by constant factors)
(oh hey that's the definition of a linear transformation)

Some rules to know by heart:
$\int{u^{n}du} = \frac{u^{n+1}}{(n+1)} + C$\\
$\int{\frac{du}{u}} = \ln{|u|} + C$\\
Learn all the trig derivatives so you can spot them and "undo" them.

\section{Definite Integrals}%%%%%%%%%%%%%%%%%%%%%%%%%%%%%%%%%%%%%%%%%%%%%%%%%%%%%%%%%%%%%%%%%%%%%%%%%%%%%%%%%%%%%%%%%%%%%%%%%%%%%%%%%%%%%%%%%%%%%%%%%%%%%%%%
Definition

\section{Integration by Substitution}%%%%%%%%%%%%%%%%%%%%%%%%%%%%%%%%%%%%%%%%%%%%%%%%%%%%%%%%%%%%%%%%%%%%%%%%%%%%%%%%%%%%%%%%%%%%%%%%%%%%%%%%%%%%%%%%%%%%%%%%%%%%%%%%
let u = ...
du = ...

\section{Integration by Parts}
let u = ... dv =...
du = ... v =...

\section{Integrating the Product of Trigonometric Functions}%%%%%%%%%%%%%%%%%%%%%%%%%%%%%%%%%%%%%%%%%%%%%%%%%%%%%%%%%%%%%%%%%%%%%%%%%%%%%%%%%%%%%%%%%%%%%%%%%%%%%%%%%%%%%%%%%%%%%%%%%%%%%%%%

$\sin^k\theta$ \\
$\cos^k\theta$ \\
$\sin^j\theta\cos^k\theta$ \\
$\tan^j\theta\sec^k\theta$ \\
$\sin(a\theta)\cos(b\theta)$ \\


\section{Integrating by Substituting Trigonometric Functions}%%%%%%%%%%%%%%%%%%%%%%%%%%%%%%%%%%%%%%%%%%%%%%%%%%%%%%%%%%%%%%%%%%%%%%%%%%%%%%%%%%%%%%%%%%%%%%%%%%%%%%%%%%%%%%%%%%%%%%%%%%%%%%%%

$\sqrt{u^2 + a^2}$\\
$\sqrt{u^2 - a^2}$\\
$\sqrt{a^2 - u^2}$\\


\section{Integrating Partial Fractions}%%%%%%%%%%%%%%%%%%%%%%%%%%%%%%%%%%%%%%%%%%%%%%%%%%%%%%%%%%%%%%%%%%%%%%%%%%%%%%%%%%%%%%%%%%%%%%%%%%%%%%%%%%%%%%%%%%%%%%%%%%%%%%%%
large/small vs small/large\\
long division\\
product of degree ones\\
product of degree ones, some raised to a higher power\\
product of non-degree ones\\

\section{Improper Integrals}%%%%%%%%%%%%%%%%%%%%%%%%%%%%%%%%%%%%%%%%%%%%%%%%%%%%%%%%%%%%%%%%%%%%%%%%%%%%%%%%%%%%%%%%%%%%%%%%%%%%%%%%%%%%%%%%%%%%%%%%%%%%%%%%
Integrals what got infinities in their limits and discontinuities up ins their ranges

\chapter{Parametric Equations}

\section{General}%%%%%%%%%%%%%%%%%%%%%%%%%%%%%%%%%%%%%%%%%%%%%%%%%%%%%%%%%%%%%%%%%%%%%%%%%%%%%%%%%%%%%%%%%%%%%%%%%%%%%%%%%%%%%%%%%%%%%%%%%%%%%%%%


\section{Polar Coordinates}%%%%%%%%%%%%%%%%%%%%%%%%%%%%%%%%%%%%%%%%%%%%%%%%%%%%%%%%%%%%%%%%%%%%%%%%%%%%%%%%%%%%%%%%%%%%%%%%%%%%%%%%%%%%%%%%%%%%%%%%%%%%%%%%



\end{document}
